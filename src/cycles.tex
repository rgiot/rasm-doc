%Table du nombre de Cycles des opcodes
% cf http://www.cpctech.org.uk/docs/instrtim.html

\section{\xlang{Durée des opcodes du Z80 sur CPC}{Z80 Opcodes Duration on CPC}}
\xlang{
La table suivante indique la durée des instructions du Z80, exprimée en nombre de NOPs équivalents, ce qui est spécifique au CPC. Par exemple, ADD A,(HL) a la même durée que 2 NOPs.
}{This table shows the duration of all z80 opcodes, expressed in number of equivalent NOPs. This is valid for CPC only. For example, ADD A,(HL) has the same duration as 2 NOPs.
}

\begin{itemize}
\item r: 8 bits register (A,B,C,D,E,H,L)
\item rr: 16 bits register (AF,BC,DE,HL,SP)
\item d: 8 bits data (0..255)
\item dd: 16 bits data
\item b: bit (0..7)
\item cond: Condition (CC,Z,M,NC,NZ,P,PO,PE)
\end{itemize}

\xlang{Les instructions manipulant le registre IY sont équivalentes à celles qui utilisent IX, elles ne sont pas indiquées ici.}{Instruction using IY register have exactly the same duration as IX instructions, they are omitted in the table.}

\renewcommand{\arraystretch}{1.0}
\setlength{\tabcolsep}{0.2em}
\begin{ttfamily}

\begin{multicols}{3}

\begin{tabular}{lc}
Opcode & \xlang{Durée}{Duration} \\
\hline
ADC r,r       & 1 \\
ADC A,(HL)    & 2 \\
ADC A,n       & 2 \\
ADC HL,rr     & 4 \\
ADC HL,SP     & 4 \\ %
ADC A,IXH     & 2 \\
ADC A,IXL     & 2 \\ %
ADC A,(IX+d)  & 3 \\
%ADC A,IYH     & 2 \\
%ADC A,IYL     & 2 \\
%ADC A,(IY+d)  & 3 \\

ADD r,r      & 1 \\
ADD A,(HL)   & 2 \\
ADD A,d      & 2 \\
ADD HL,dd    & 3 \\
ADD IX,rr    & 4 \\
%ADD IY,rr    & 4 \\
%ADD IY,IY   & 4 \\
ADD IX,IX    & 4 \\
ADD IX,SP    & 4 \\
%ADD IY,SP    & 4 \\
ADD A,IXH    & 2 \\
ADD A,IXL    & 2 \\
%ADD A,IYH   & 2 \\
%ADD A,IYL   & 2 \\
ADD A,(IX+d) & 3 \\
%ADD A,(IY+d) 3 \\

AND r       & 1 \\
AND d       & 2 \\
AND (HL)    & 2 \\
AND IXH     & 2 \\
AND IXL     & 2 \\
%AND IYH    & 2 \\
%AND IYL    & 2 \\
AND (IX+d)  & 3 \\
%AND (IY+d) & 3 \\

BIT r        & 2 \\
BIT (HL)     & 3 \\
BIT b,(IX+d) & 6 \\
BIT b,(IX+d),r & 6 \\

CALL dd     & 5 \\
CALL cond,dd  & 5 / 3 \\
CCF       & 1 \\
CP r      & 1 \\
CP d      & 2 \\
CP (HL)   & 2 \\
CP IXH    & 2 \\



\end{tabular}

\columnbreak

\begin{tabular}{lc}
Opcode & \xlang{Durée}{Duration} \\
\hline

CP IXL    & 2 \\
CP (IX+d) & 3 \\
CPD       & 4 \\
CPDR      & 6 / 4 \\
CPI       & 4 \\
CPIR      & 6 / 4 \\
CPL       & 1 \\

DAA        & 1\\
DEC r      & 1\\
DEC rr     & 2\\
DEC (HL)   & 3\\
DEC IX     & 3\\
DEC IXH    & 2\\
DEC IXL    & 2\\
DEC (IX+d) & 6\\

DI          & 1 \\
DJNZ d      & 4 / 3 \\

EI          & 1 \\
EX AF,AF’   & 1 \\
EX DE,HL    & 1 \\
EX (SP),HL  & 6 \\
EX (SP),IX  & 7 \\
EXX         & 1 \\
HALT        & 1 \\

IM 0 & 2 \\
IM 1 & 2 \\
IM 2 & 2 \\

IN A,(d)   & 3 \\
IN r,(C)   & 4 \\
INC r      & 1 \\
INC rr     & 2 \\
INC (HL)   & 3 \\
INC IX     & 3 \\
INC IXH    & 2 \\
INC IXL    & 2 \\

\end{tabular}

\columnbreak

\begin{tabular}{lc}
Opcode & \xlang{Durée}{Duration} \\
\hline

INC (IX+d) & 6 \\
IND        & 5 \\
INI        & 5 \\
INIR       & 6 / 5 \\
INDR       & 6 / 5 \\

JP dd       & 3 \\
JP cond,dd     & 3 \\
JP (HL)     & 1 \\
JP (IX)     & 2 \\

JR d      & 3 \\
JR C,d    & 3 / 2 \\
JR NC,d   & 3 / 2 \\
JR NZ,d   & 3 / 2 \\
JR Z,d    & 3 / 2 \\

LD r,r            & 1 \\
LD r,d            & 2 \\
LD A,(rr)         & 2 \\
LD r,(HL)         & 2 \\
LD (rr),A         & 2 \\
LD SP,HL          & 2 \\
LD r,IXH          & 2 \\
LD r,IXL          & 2 \\
LD SP,IX          & 3 \\
LD rr,dd          & 3 \\
LD (HL),d         & 3 \\
LD A,R            & 3 \\
LD R,A            & 3 \\
LD A,I            & 3 \\
LD I,A            & 3 \\
LD IXH,d          & 3 \\
LD IXL,d          & 3 \\
LD A,(dd)         & 4 \\
LD (dd),A         & 4 \\
LD IX,dd          & 4 \\
LD HL,(dd)        & 5 \\

\end{tabular}

\columnbreak

\begin{tabular}{lc}
Opcode & \xlang{Durée}{Duration} \\
\hline
LD BC,(dd)        & 6 \\
LD DE,(dd)        & 6 \\
LD (dd),HL        & 5 \\
LD r,(IX+d)       & 5 \\
LD (dd),rr        & 6 \\
LD IX,(dd)        & 6 \\
LD (dd),IX        & 6 \\
LD (IX+d),d       & 6 \\
LD (dd),SP        & 6 \\

LD SP,(dd)        & 6 \\

LDD     & 5 \\
LDI     & 5 \\
LDDR    & 6 / 5 \\
LDIR    & 6 / 5 \\

NEG & 2 \\
NOP & 1\\

OR r      & 1 \\
OR d      & 2 \\
OR (HL)   & 2 \\
OR IXH    & 2 \\
OR IXL    & 2 \\
OR (IX+d) & 3 \\

OUT (d),A & 3 \\
OUT (C),r & 4 \\
OUT (C),0 & 4 \\

OUTD      & 5     \\
OUTI      & 5     \\
OTDR      & 6 / 5 \\
OTIR      & 6 / 5   \\
POP rr    & 3 \\
POP IX    & 4 \\
PUSH rr   & 4 \\
PUSH IX   & 5 \\

RES b,r       & 2 \\
RES b, (HL)    & 4 \\
RES b,(IX+d)  & 7 \\
RES b,(IX+d),r & 7 \\

\end{tabular}

\columnbreak

\begin{tabular}{lc}
Opcode & \xlang{Durée}{Duration} \\
\hline

RET     & 3    \\
RET cond   & 4 / 2 \\
RETN & 4 \\
RETI & 4 \\

RL r        & 2 \\
RL (HL)     & 4 \\
RL (IX+d)   & 7 \\
RL (IX+d),r    & 7 \\

RLC r       & 2 \\
RLC (HL)    & 4 \\
RLC (IX+d)  & 7 \\
RLC (IX+d),r   & 7 \\

RLCA        & 1 \\
RLA         & 1 \\
RLD         & 5 \\
RR r       & 2 \\
RR (HL)    & 4 \\
RR (IX+d)   & 7 \\
RR (IX+d),r    & 7 \\
RRA         & 1 \\
RRC r       & 2 \\
RRC (HL)    & 4 \\
RRC (IX+d)  & 7 \\
RRC (IX+d),r   & 7 \\

RRD         & 5 \\
RRCA        & 1 \\
RST d       & 4 \\

SBC A,r      & 1 \\
SBC A,d      & 2 \\
SBC A,IXH    & 2 \\
SBC A,IXL    & 2 \\
SBC A,(HL)   & 2 \\
SBC A,(IX+d) & 3 \\
SBC HL,rr    & 4 \\
SBC HL,SP    & 4 \\

\end{tabular}

\columnbreak

\begin{tabular}{lc}
Opcode & \xlang{Durée}{Duration} \\
\hline
SCF           & 1 \\
SET b,r       & 2 \\
SET b, (HL)    & 4 \\
SET b,(IX+d)  & 7 \\
SET b,(IX+d),r & 7 \\
SLA r       & 2 \\
SLA (HL)    & 4 \\
SLA (IX+d)    & 7 \\
SLA (IX+d),r   & 7 \\

SLL r       & 2 \\
SLL (HL)    & 4 \\
SLL (IX+d)    & 7 \\
SLL (IX+d),r   & 7 \\

SRA r       & 2 \\
SRA (HL)    & 4 \\
SRA (IX+d)    & 7 \\
SRA (IX+d),r   & 7 \\

SRL r       & 2 \\
SRL (HL)    & 4 \\
SRL (IX+d)    & 7 \\
SRL (IX+d),r   & 7 \\

SUB r         & 1 \\
SUB d         & 2 \\
SUB (HL) & 2\\
SUB IXH & 2 \\
SUB IXL & 2 \\
SUB (IX+d) & 3 \\
%SET b,(IY+d) & 7 \\
%SLA (IY+d) & 7 \\
%SLL (IY+d) & 7 \\
%SRA (IY+d) & 7 \\ \\
%SRL (IY+d) & 7 \\

XOR r       & 1 \\
XOR d       & 2 \\
XOR (HL)    & 2 \\
XOR IXH     & 2 \\
XOR IXL     & 2 \\
XOR (IX+d)  & 3 \\

\end{tabular}



\end{multicols}
\end{ttfamily}
